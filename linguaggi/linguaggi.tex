\documentclass{article}
\usepackage[utf8]{inputenc}

\title{Linguaggi di programmazione}
\author{kocierik }
\date{September 2021}

\begin{document}

\maketitle
\tableofcontents

\section{Tipi di linguaggi }
\subsection{Linguaggi imperativi e dichiarativi}
I linguaggi \textbf{imperativi} basati sullo stato (variabili, assegnamento).\\
I linguaggi \textbf{dichiarativi} sono basati sulla nozione di funzione ((fun(x).X+2)3;)\\
I linguaggi\textbf{funzionali} (haskell) sono basati sulla nozione di stato: risultato del programma\\
I linguaggi \textbf{logici} (prolog), sono basati sulle relazioni del risultato di un programma

\subsection{Confrontare linguaggi}
\begin{itemize}
    \item Caratteristiche intrinsceche (sintassi)
    \item Espressività (complessità)
    \item Didattica (difficoltà apprendimento)
    \item Leggibilità (lettura programma)
    \item Robustezza (tipi, ecc)
    \item Generalità
    \item Efficienza (velocità)
    \item Integrabilità
\end{itemize}

\subsection{Macchine astratte}
Il ciclo di istruzioni di una macchina fisica è il classico \textbf{fetch-decode-execute}\\
Una macchina astratta invece ha come componente fondamentale l'interprete
\subsubsection{Interprete}
\begin{itemize}
    \item Elaborazione \textbf{tipi primitivi}
    \item operazioni per il controllo della \textbf{sequenza di esecuzione}
    \item Operazione per il \textbf{controllo del trasferimento}
    \item Operazione per la \textbf{gestione della memoria}
\end{itemize}

\subsubsection{Linguaggi macchina}
\begin{itemize}
    \item \textbf{M} macchina astratta
    \item $L_M$ Linguaggio macchina \textbf{M}
    \item $L_M$ è il linguaggio compreso dall'interprete \textbf{M}
\end{itemize}

Le istruzioni implementate dall'ALU, possono essere:
\begin{itemize}
    \item \textbf{RISC} Complex Instruction Set Computers
    \item \textbf{CISC} Reduced Instruction Set Computers
\end{itemize}

\subsection{Implementazione interpretativa pura}
$P^L_r$ indica un programma scritto nel linguaggio \textbf{L}
$M_L$ è realizzato scrivendo un \textbf{interprete} per $L$ su $Mo_L_o$ \\
Un interprete per il linguaggio $L$ scritto nel linguaggio $L_o$ è un programma che realizza una funzione parziale.
Ovvero l’interprete "\textbf{calcola} la corretta semantica" del programma!

\subsection{Implementazione compilativa pura}
I programmi in \textbf{L} sono tradotti in programmi equivalenti in $L_o$\\
La traduzione è fatto dal compilatore, il quale si occuperà di "tradurre" il codice rendendolo leggibile alla macchina, preservando la semantica.

$I^L^o_L_1 =$ un interprete scritto in $L0$ che esegue programmi scritti in $L1$
$I^L^0_L_1(P^L^1,x) =$ risultato del calcolo del programma P (scritto in L1) con x come dato in input


\end{document}

