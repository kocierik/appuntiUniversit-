\documentclass[12pt]{article}
\usepackage{amsfonts, amssymb, amsmath}
\parindent 0px

\begin{document}

\title{Algebra e geometria}
\author{Koci Erik}
\maketitle

\section{Spazi vettoriali}

Uno \textbf{spazio vettoriale} su un campo $\kappa$ è un insieme $V$ in cui sono definite: \\
\begin{enumerate}
    \item Un'operazione interna tra elementi di $V$, detta \textbf{somma}
    \item Un'operazione di \textbf{prodotto} che associa ad ogni coppia formata da un elemento di $V$ (vettore) e da un elemento di $\kappa$ (scalare) un (unico) elemento di $V$. 
    \item Devono rispettare le proprietà dei numeri $reali$
\end{enumerate}

L'insieme $R\textsuperscript{$n$} = \{(x\textsubscript{1},x\textsubscript{2},x\textsubscript{n}):x\textsubscript{i}\in R\}$ di tutte le n-uple ordinate di numeri reali con le due operazioni di \textbf{somma} e \textbf{moltiplicazione per uno scalare} così definite:
$$(x\textsubscript{1},x\textsubscript{2},x\textsubscript{n})+(y\textsubscript{1},y\textsubscript{2},y\textsubscript{n})=
(x\textsubscript{1}+y\textsubscript{1},x\textsubscript{2}+y\textsubscript{2},x\textsubscript{n}+y\textsubscript{n})$$
$$\lambda(x\textsubscript{1}+x\textsubscript{2}+x\textsubscript{n})=
(\lambda k\textsubscript{1}+\lambda k\textsubscript{2}+\lambda k\textsubscript{n})\hspace{1cm} \lambda \in R$$
E' uno \textbf{spazio vettoriale su $R$}\\\\
In particolare
\begin{itemize}
    \item Il vettore nullo di cui si parla è $0=(0,0,..,0)$
    \item L'opposto di $v=(x\textsubscript{1},x\textsubscript{2},x\textsubscript{n})$ è $-v=(-x\textsubscript{1},-x\textsubscript{2},-x\textsubscript{n})$
    \item L'insieme dei vettori nel piano e dei vettori nello spazio possono essere identificati, rispettivamente, con $R\textsubscript{2}$ e $R\textsubscript{3}$
\end{itemize}
\section{Sottospazi vettoriali}
 Sia $V$ uno spazio vettoriale su un campo $\kappa$. Un sottoinsieme $W\subsetneq V$ è un \textbf{sottospazio vettoriale} di $V$ se valgono le seguenti proprietà:
\begin{enumerate}
    \item Sommate due coppie di elementi il risultato deve ancora \textbf{appartenere} all'insieme definito.
    $$W\textsubscript{1}+W\textsubscript{2}\in W$$
    \item Moltiplicati due valori per una costante, il risultato deve ancora \textbf{appartenere} all'insieme.
    $$\lambda w \in W$$
    \item $W$ \textbf{deve contenere lo 0} e l'opposto di ogni suo elemento.
\end{enumerate}
\textbf{Esempio:} \\\\
Stabilire se è un sottospazio vettoriale di $V=R\textsuperscript{2}$
$$W = \{(x,y)\in R\textsuperscript{2}: 2y-3x=0\}$$
Considero 2 generici vettori e verifico che la loro somma appartiene a sua volta a $W$:
$$W\textsubscript{1}+W\textsubscript{2}=(x\textsubscript{1},y\textsubscript{1})+(x\textsubscript{2},y\textsubscript{2})=(x\textsubscript{1}+x\textsubscript{2},y\textsubscript{1}+y\textsubscript{2})$$
Considero 2 generici vettori e verifico che appartengono a sua volta a $W$:
$$2 (y\textsubscript{1}+y\textsubscript{2})-3(x\textsubscript{1}+x\textsubscript{2})=2y\textsubscript{1}-3x\textsubscript{1}+2y\textsubscript{2}-3x\textsubscript{2}=0$$
Questa proprietà è verificata perchè sappiamo per definizione che $W\textsubscript{1} \in W$ e $W\textsubscript{2} \in W$.
\\\\
Considero 2 generici vettori e verifico che il prodotto da come risultato un valore che appartiene a sua volta a $W$:
$$\lambda w\textsubscript{1}=\lambda (x\textsubscript{1},y\textsubscript{1})=(\lambda x\textsubscript{1},\lambda  y\textsubscript{1})$$
$$2\lambda y\textsubscript{1}-3 \lambda k\textsubscript{1} = \lambda(2y\textsubscript{1}-3x\textsubscript{1})=0$$ \\
Anche questa proprietà è stata verificata perchè sappiamo per definizione che $W\textsubscript{1} \in W$.\\\\
Ricordarsi sempre di verificare se esiste il vettore \textbf{nullo!}

\section{Vettori linearmente dipendenti-indipendenti}
Sia $V$ uno spazio vettoriale su un campo $\kappa$ e siano $\lambda \textsubscript{1},\lambda \textsubscript{2},\lambda \textsubscript{n}$ una qualunque somma del tipo: $$\lambda\textsubscript{1}V\textsubscript{1}+\lambda\textsubscript{2}V\textsubscript{2}+\lambda\textsubscript{n}V\textsubscript{n}$$
\\
Si dice che i vettori $V\textsubscript{1},V\textsubscript{2},V\textsubscript{n}$ sono \textbf{linearmente dipendenti} se esiste una loro combinazione lineare, a coefficienti non tutti nulli, che dà come risultato il vettore nullo. \\
Si dice che i vettori $V\textsubscript{1},V\textsubscript{2},V\textsubscript{n}$ sono \textbf{linearmente indipendenti} se l'unica loro combinazione lineare che dà come risultato il vettore nullo è quella con tutti i coefficienti nulli, ovvero se: 
$$\lambda\textsubscript{1}V\textsubscript{1}+\lambda\textsubscript{2} V\textsubscript{2}+\lambda\textsubscript{n} V\textsubscript{n}=0 \implies \lambda\textsubscript{1}=\lambda\textsubscript{2}=\lambda\textsubscript{n}=0$$ 
\textbf{Esempio}
Verificare se i seguenti vettori $\in R\textsuperscript{2}$ sono linerarmente dipendenti o indipendenti:
$$V\textsubscript{1}=(1,2) \hspace{1cm} V\textsubscript{2}=(4,1)$$
Per scoprirlo devo \textbf{risolvere:} $$\lambda \textsubscript{1}V\textsubscript{1}+\lambda \textsubscript{2}V\textsubscript{2}=0$$
$$\lambda\textsubscript{1}(1,2)+\lambda\textsubscript{2}(4,1)=(0,0) \implies (\lambda\textsubscript{1},2\lambda\textsubscript{1})+(4\lambda\textsubscript{1},\lambda\textsubscript{1})=(0,0)$$
$$(\lambda\textsubscript{1}+4\lambda\textsubscript{2},2\lambda\textsubscript{1}+ \lambda\textsubscript{2})=(0,0)$$
$$
\begin{cases}
  \lambda\textsubscript{1}+4\lambda\textsubscript{2}=0 \\
       2\lambda\textsubscript{1}+\lambda\textsubscript{2}=0 
\end{cases}$$
       
$$\begin{cases}
  \lambda\textsubscript{1}=-4\lambda\textsubscript{2} \\
    2(-4\lambda\textsubscript{2})+\lambda\textsubscript{2}=0 
\end{cases}$$
$$\begin{cases}
  \lambda\textsubscript{1}=0 \\
    \lambda\textsubscript{2}=0 
\end{cases}$$
Due vettori sono \textbf{linearmente dipendenti} quando \textbf{stanno sulla stessa retta} ovvero quando $V\textsubscript{1} = \lambda V \textsubscript{2}$. \\\\ Nel caso in cui una soluzione del sistema dia come risultato $0=0$ significa che il sistema ha infinite soluzioni. Di conseguenza possiamo prendere un \textbf{qualunque} valore. Se i vettori stanno sullo stesso piano allora sono \textbf{linearmente indipendenti}. 


\section{Insiemi di generatori e basi}
Dati $n$ vettori $V\textsubscript{1},V\textsubscript{2},V\textsubscript{n}$ di uno spazio vettoriale $V$ su un campo $\kappa$ si definisce \textbf{span lineare} l'insieme:
$$Span(V\textsubscript{1},V\textsubscript{2},V\textsubscript{n})=\{\lambda\textsubscript{1}V\textsubscript{1}+\lambda\textsubscript{2}V\textsubscript{2}+\lambda\textsubscript{n}V\textsubscript{n} | \lambda\textsubscript{i} \in \kappa\}$$
Di tutti i vettori che si possono scrivere come combinazione lineare di $V\textsubscript{1},V\textsubscript{2}, V\textsubscript{n}$. Lo span è l'insieme di tutte le combinazioni lineari che possiamo fare da V\textsubscript{1} a V\textsubscript{n}. Esso può essere indicato con due principali sintassi: 
$$ L(V\textsubscript{1}.,V\textsubscript{2},V\textsubscript{n}) \hspace{1cm}
<V\textsubscript{1},V\textsubscript{2},V\textsubscript{n}>$$
\textbf{Esempio}: Se $V\textsubscript{1}\neq 0$ è un vettore di $R\textsuperscript{2}$ allora $L(V\textsubscript{1})$ contiene tutti i vettori del tipo $\lambda\textsubscript{1}V\textsubscript{1}$. Si tratta quindi della retta passante per l'origine di direzione $V\textsubscript{1}$. Essa comprende tutti i suoi multipli. 
\begin{itemize}
    \item sono linearmente indipendenti
    \item sono un sistema di generatori
\end{itemize}
Si dimostra che, se $V$ ha una base costituita da $n$ vettori, ogni altra base di $V$ è costituita da $n$ vettori. Si dice allora che $V$ ha \textbf{dimensione} $n$ e si scrive $dimV=n$.\\
La scomposizione di un vettore $V$ come combinazione lineare dei vettori di una base è unica, ovvero se $V\textsubscript{1},V\textsubscript{2},V\textsubscript{n}$ sono una base dello spazio vettoriale "v
 allora per ogni $v \in V$ esistono unici $\lambda\textsubscript{1},\lambda\textsubscript{2},\lambda\textsubscript{n}$ tali che: 
 $$V=\lambda\textsubscript{1}V\textsubscript{1}+\lambda\textsubscript{2}V\textsubscript{2}+\lambda\textsubscript{n}V\textsubscript{n}$$
 I coefficienti di $\lambda\textsubscript{1},\lambda\textsubscript{2},\lambda\textsubscript{n}$ si dicono \textbf{componenti} di $V$ rispetto alla base $V\textsubscript{1},V\textsubscript{2},V\textsubscript{n}$.\\\\
 \textbf{Esempio:} \\\\
 In $R\textsuperscript{2}$ una possibile base è $V\textsubscript{1}=(1,0)$ e $V\textsubscript{2}=(0,1)$. infatti:
\begin{itemize}
    \item sono linearmente indipendenti (non sono paralleli)
    \item sono generatori infatti per ogni generico $V=(a,b)$ di $R\textsuperscript{2}$ vale la relazione $V=a(1,0)+b(0,1)$
\end{itemize}
 \section{Risoluzioni esercizi}
Per verificare se due vettori sono linearmente dipendenti o indipendenti devo:
\begin{itemize}
    \item risolvere $\lambda\textsubscript{1}V\textsubscript{1}+\lambda\textsubscript{2}V\textsubscript{2}=0$, e quindi convincermi che l'unica soluzione è $\lambda\textsubscript{1}=\lambda\textsubscript{2}=0$. Oppure controllare se i \textbf{vettori sono multipli}, se sono multipli tra loro, allora i vettori sono dipendenti.
\end{itemize}

\end{document}

