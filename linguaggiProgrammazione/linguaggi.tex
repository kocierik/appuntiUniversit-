\documentclass{article}
\usepackage[utf8]{inputenc}
\usepackage{eucal}

\title{Linguaggi di programmazione}
\author{kocierik }
\date{September 2021}

\begin{document}
\tableofcontents

\maketitle

\section{Tipi di linguaggi }
\subsection{Linguaggi imperativi e dichiarativi}
I linguaggi \textbf{imperativi} basati sullo stato (variabili, assegnamento).\\
I linguaggi \textbf{dichiarativi} sono basati sulla nozione di funzione ((fun(x).X+2)3;)\\
I linguaggi\textbf{funzionali} (haskell) sono basati sulla nozione di stato: risultato del programma\\
I linguaggi \textbf{logici} (prolog), sono basati sulle relazioni del risultato di un programma

\subsection{Confrontare linguaggi}
\begin{itemize}
    \item Caratteristiche intrinsceche (sintassi)
    \item Espressivit� (complessit�)
    \item Didattica (difficolt� apprendimento)
    \item Leggibilit� (lettura programma)
    \item Robustezza (tipi, ecc)
    \item Generalit�
    \item Efficienza (velocit�)
    \item Integrabilit�
\end{itemize}

\subsection{Macchine astratte}
Il ciclo di istruzioni di una macchina fisica � il classico \textbf{fetch-decode-execute}\\
Una macchina astratta invece ha come componente fondamentale l'interprete
\subsubsection{Interprete}
\begin{itemize}
    \item Elaborazione \textbf{tipi primitivi}
    \item operazioni per il controllo della \textbf{sequenza di esecuzione}
    \item Operazione per il \textbf{controllo del trasferimento}
    \item Operazione per la \textbf{gestione della memoria}
\end{itemize}

\subsubsection{Linguaggi macchina}
\begin{itemize}
    \item \textbf{M} macchina astratta
    \item $L_M$ Linguaggio macchina \textbf{M}
    \item $L_M$ � il linguaggio compreso dall'interprete \textbf{M}
\end{itemize}

Le istruzioni implementate dall'ALU, possono essere:
\begin{itemize}
    \item \textbf{RISC} Complex Instruction Set Computers
    \item \textbf{CISC} Reduced Instruction Set Computers
\end{itemize}

\subsection{Implementazione interpretativa pura}
$P^L_r$ indica un programma scritto nel linguaggio \textbf{L}
$M_L$ � realizzato scrivendo un \textbf{interprete} per $L$ su $Mo_L_o$ \\
Un interprete per il linguaggio $L$ scritto nel linguaggio $L_o$ � un programma che realizza una funzione parziale.
Ovvero l�interprete "\textbf{calcola} la corretta semantica" del programma!\\
Questo tipo di compilazione ha \textbf{scarsa efficienza}, ai tempi di esecuzione vanno aggiunti i tempi di alla decodifica. In questo modo \textbf{ogni volta} che eseguo una istruzione dovr� essere \textbf{tradotta}. Al contrario della \textbf{compilazione pura} dove dovr� eseguire il codice \textbf{una sola volta}.
\\\\
$I^L^0_L_1(P^L^1,x) =$ risultato del calcolo del programma P (scritto in L1) con x come dato in input.\\\\


\subsection{Implementazione compilativa pura}
I programmi in \textbf{L} sono tradotti in programmi equivalenti in $L_o$\\
La traduzione � fatto dal compilatore, il quale si occuper� di "tradurre" il codice rendendolo leggibile alla macchina, preservando la semantica.
\\\\
$I^L^o_L_1 =$ un interprete \textbf{scritto} in $\textbf{L0}$ che esegue programmi \textbf{scritti} in $L1$\\\\

Questo tipo di implementazione risulta difficile data la lontananza tra $L$ e $L_0$, ma ha una buona efficienza. Un esempio di linguaggio compilativo � \textbf{C}, il quale rende il codeice leggibile alla macchina.\\

\subsection{Come generare un compilatore}
\begin{itemize}
    \item E' possibile utilizzare strumenti automatici:
    \begin{itemize}
        \item \textbf{Lex} generatore di analizzatori \textbf{lessicali}
        \item \textbf{Yacc} generatore di analizzatori \textbf{sintattici}
    \end{itemize}
    \item Implementazione via kernel
    \begin{itemize}
        \item  per implementare $L$ si scrive un interprete o un compilatore in $H$ che � un sottoinsieme ristretto di $L$.
    \end{itemize}
\end{itemize}

\subsubsection{Conclusioni}
In linea di principio:
\begin{itemize}
    \item Buona flessibilit�
    \item Pi� facile da realizzare
    \item Occupa meno memoria
\end{itemize}
Soluzione di tipo compilativo
\begin{itemize}
    \item privilegia l�efficienza
    \item soluzione di tipo interpretativo
    \item privilegia flessibilit� e portabilit�
\end{itemize}

Per realizzare un interprete o un compilatore � necessario che siano pienamente espressivi e quindi di implementare almeno tutte le funzioni matematiche basilari.
Ne parleremo in seguito con il concetto di \textbf{Turing-completezza}.

\section{Descrivere un linguaggio}
La descrizione di un linguaggio avviene su 3 dimensioni:
\begin{itemize}
    \item \textbf{Sintassi} (regole di formazione, significato)
    \item \textbf{Semantica} (attribuzione di significato)
    \item \textbf{Pragmatica} (in quale modo frasi corrette e sensate sono usate, quando usare determinati costrutti)
    \item \textbf{Implementazione} (come eseguire una frase corretta rispettandone la semantica)
\end{itemize}

\subsubsection{Sintassi}
\begin{itemize}
    \item \textbf{Aspetto lessicale} (parole che si possono utilizzare)
    \begin{itemize}
        \item \textbf{Descrizione del lessico}
        \begin{itemize}
            \item Dizionari
            \item strutture complesse
        \end{itemize}
        \item \textbf{Errore} (vocaboli inconsistente)
    \end{itemize}
    \item \textbf{Aspetto grammaticale} (frasi corrette che si possono costruire con il lessico)
    \begin{itemize}
        \item Descrizione attraverso regole grammaticali
        \item \textbf{Errore} (frasi scorrette)
    \end{itemize}
\end{itemize}

\subsubsection{Semantica}
\begin{itemize}
    \item Per il \textbf{lessico} =  dizionario
    \item Per le frasi devo sapere:
    \begin{itemize}
        \item A quale \textbf{linguaggio} la frase appartiene
        \item Su quale linguaggio basarmi per dare \textbf{significato}
    \end{itemize}
\end{itemize}
\subsubsection{Pragmatica}
\begin{itemize}
    \item Insieme di regole che guidano l'uso
\end{itemize}
\subsubsection{Implementazione}
Eseguire una frase sintatticamente corretta rispettando la semantica

\subsection{Lessico e frasi un linguaggio}
\begin{itemize}
    \item \textbf{Alfabeto}
    \begin{itemize}
        \item \textbf{Lessico} ( insieme di parole di sequenze finite (parole) costruite con caratteri
        \begin{itemize}
            \item \textbf{Frasi} (insieme di sequenze finite costruite con parole del lessico
        \end{itemize}
    \end{itemize}
\end{itemize}
\subsubsection{Linguaggio formale}
\begin{itemize}
    \item L'alfabeto � un insieme di simboli
    \item Una parola � sequenza di simboli
    \item Un linguaggio formale � un insieme di parole
\end{itemize}
\begin{itemize}
    \item Per indicare una parola vuota utilizzo questa sintassi:
$$A^0=\{\epsilon\}$$
    \item Contiene solo parole di lunghezza 1
$$A^1=\{\epsilon\}$$
    \item Contiene parole di lunghezza 2
    $$A^1=\{\epsilon\}$$
    \item Contiene tutte le parole
    $$A^*$$
\end{itemize}
$N^K$ � numerabile
$N^*$ � numerabile

\subsection{Notazioni e definizioni ausiliari}
\begin{itemize}
    \item \textbf{Lunghezza} di una parola
    $$|\epsilon|=0$$
    $$|aw|=1+|w|$$
    \item \textbf{Concatenazione} $x$ concatenata con $y$, $xy$ � la parola ottenuta concatenando
    \item \textbf{Leggi della concatenazione} Vale l'associativit�
    \item \textbf{Sottostringa V di una stringa w}
    \begin{itemize}
        \item Ogni stringa � sottostringa di se stessa
        \item $\epsilon$ � sottostringa di ogni stringa
    \end{itemize}
    \item \textbf{Suffisso:} $v$ � suffisso di $w$ sse $\exists x \in A^*. w=xv$
    \item \textbf{Prefisso:} $v$ � prefisso di  $w$ sse $\exists x \in A^*$. $w=vx$
    \item \textbf{Potenza ennesima} valgono le sequenze ripetute $W^n^+^1 = ww^n$
    \item \textbf{Linguaggio L su alfabeto A} dove $L$ � sottoinsieme di $A^*$
\end{itemize}





\end{document}

