\documentclass{article}
\usepackage[utf8]{inputenc}

\title{Reti di calcolatori}
\author{kocierik }
\date{September 2021}

\begin{document}

\tableofcontents

\maketitle

\section{Rete di calcolatori}
Una rete di calcolatori � un insieme di dispositivi autonomi e interconnessi
Utilizzata per condividere risorse, accessi remoti...

\subsection{Classificazione reti}
\begin{itemize}
    \item \textbf{PAN} Persona Area Network
    \item \textbf{LAN} Local Area Network
    \item \textbf{MAN} Metropolitana Area Network
    \item \textbf{WAN} Wide Area Network
    \item \textbf{Internet} rete globale
\end{itemize}

\subsubsection{Prestazioni}
L'\textbf{indice nominale massimo} indica la massima velocit� raggiungibile.
Le prestazioni delle reti di calcolatori vengono misurate attraverso:
\begin{itemize}
    \item \textbf{Capacit� di trasmissione}: (numero di bit o in byte
    \item \textbf{Ritardo del collegamento}: tempo richiesto ai dati per transitare da mittente a destinatario
\end{itemize}
Il \textbf{jitter} indica di variazione del ritardo di rete di una o pi� caratteristiche di un segnaler . Differenza di tempo tra i pacchetti inviati.

\subsection{Componenti}
\begin{itemize}
    \item \textbf{Scheda di rete}, codifica e trasmette i dati (driver necessari API)
    \item \textbf{Mezzo di trasmissione}, supporto fisico (doppini,fibra)
    \item \textbf{Connettore di rete}, interfaccia per il collegamento del dispositivo, \textbf{RJ-45} (cavo ethernet.
    \item \textbf{Protocolli di rete}, regole da rispettare per garantire compatibilit� tra dispositivi
\end{itemize}

\textbf{gabbia di Faraday} si intende qualunque sistema conduttore in grado d'isolare l'ambiente interno da un qualunque campo elettrostatico.\\
 \textbf{PCI} = Peripheral Component Interconnect

\subsection{Collegamenti e infrastrutture di rete}
\begin{itemize}
    \item \textbf{Connesione o collegamento di rete}
    \begin{itemize}
        \item nodi
        \item host
    \end{itemize}
    \item \textbf{Infrastruttura di rete}, struttura di rete a connesioni multiple:
    \begin{itemize}
        \item Completamente connesse (ogni nodo � raggiungibile in un passo)
        \item Parzialmente connesse ( connessione coperta solo da un singolo percorso)
        \item Partizioi di rete (gruppo di componenti isolato da altri
    \end{itemize}
    \item \textbf{Cammino dei segnali}, diretto o indiretto attraverso connessioni in sequenza
\end{itemize}


\end{document}

