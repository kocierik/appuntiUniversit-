
\documentclass{article}
\usepackage[utf8]{inputenc}
\usepackage{amsfonts, amssymb, amsmath}
\usepackage{graphicx}
\graphicspath{ {./images/} }
\parindent 0px
\title{Calcolo numerico}
\author{kocierik }
\date{September 2021}

\begin{document}

\tableofcontents

\maketitle

\section{Norma di un vettore}

Possiamo individuare 3 tipi di norme:
\begin{itemize}
    \item Norma infinito
    \item Norma 1
    \item Norma 2
\end{itemize}

Una applicazione lineare � chiamata norma quando:
\begin{itemize}
    \item $||x||>= 0$
    \item $||x||=0$ se e solo se $x=0$
    \item $||\alpha x|| = |\alpha| ||x||$ per ogni x $\in$ R
    \item Per ogni $x$ e $y$ vale la disuguaglianza triangolare:
$$ |x+y| <= |x|+|y|$$
\end{itemize}
Il numero $||x-y||$ rappresenta la distanza tra i punti $x$ e $y$
Un esempio di norma � fornito dalla norma $p$ definita per $1<=p<+\infty$\\

 Per calcolare la norma di una matrice bisogna:
 p = raggio spettrale = calcola autovalori metto valore assoluti e prendo il massimo
\begin{itemize}
    \item Norma 1 = calcolare la somma dei valori assoluti degli elementi di ogni colonna. Il massimo tra questi risultati sar� il valore della nostra norma.
    \item Norma infinito  = massimo dalla somma delle righe con modulo
    \item Norma 2 = $||A||_2 = \sqrt{p(A^TA)}
\end{itemize}
\begin{center}
\includegraphics[width=12cm]{img/norme.png}
\end{center}



\end{document}

